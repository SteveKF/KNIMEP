%% Einleitung.tex
\chapter{Einleitung}
\label{ch:Einleitung}
%% ==============================
Die Suche nach einem bestimmten Produkt auf einer Internetseite liefert meistens zu viele oder gar keine Ergebnisse.  Dies liegt daran, dass Suchabfragen auf harten Bedingungen beruhen, die den Wünschen des Kunden entsprechen. Infolgedessen müssen Produkte die Bedingungen des Kunden erfüllen, da sie ansonsten nicht in der Ergebnismenge aufgelistet werden. Dabei macht es keinen Unterschied, dass bei einem leeren Ergebnis Produkte existieren, die beinahe die Kriterien des Kunden erfüllen würden. Andererseits gelangen Produkte in die Ergebnismenge, die in allen ausgewählten Kriterien schlechter sind als andere. Das führt zu großen Ergebnismengen, die verhindert werden können. Um diese beiden Effekte, auf die später in der Arbeit noch genauer eingegangen werden, zu vermeiden, kann der User seine Wünsche in Präferenzen äußern. Präferenzen erlauben es dem User, im Gegensatz zu Bedingungen, nicht nur seine eigenen Bevorzugungen in die Suche mit einfließen zu lassen, sondern erzwingt eine reine Ausgabe der besten Ergebnisse. Präferenzen sorgen zusätzlich dafür, dass die Wünsche des Kunden nicht zwangsläufig erfüllt werden müssen, wenn es nicht möglich ist. Dadurch werden ideale Ergebnisse ausgegeben, die den Wünschen des Kunden am besten entsprechen.
Bei Präferenzen ist es daher nicht nur wichtig, welche Attribute für den Nutzer am wichtigsten sind, sondern auch wie diese bevorzugt werden. Beispielsweise können die Attribute, im Falle eines Autos, Preis und PS lauten. Der Preis ist wünschenswerter Weise so gering und die PS Anzahl so hoch wie möglich.
Die Produkte werden dann anhand dieser Präferenzen miteinander verglichen. Das Ergebnis konstituiert sich aus den  Produkten, denen die restlichen untergeordnet sind. Die Dominanz ist hierbei wie folgt definiert: Ein Produkt dominiert ein anderes, wenn es in allen Kriterien mindestens genauso gut und in einem Kriterium besser ist. Ob ein Produkt in einem Kriterium besser ist als ein anderes, hängt von den Präferenzen ab. Da solche Vergleiche mit Präferenzen nur Ergebnisse liefern, welche die Wünsche des Kunden nicht perfekt erfüllen müssen, ist erkennbar, dass sich Präferenzen wie weiche Bedingungen verhalten. Weiche Bedingungen sind somit Wünsche, die nicht zwangsläufig erfüllt werden müssen.
Ein Problem stellt sich durch die Menge der Kriterien für die Bewertung. Umso mehr relevante Kriterien in die Bewertung mit einfließen, desto schwieriger bestimmt sich eine Ergebnismenge. Zu viele Produkte sind dadurch nicht mehr vergleichbar und bleiben somit untergeordnet. Die Menge der Produkte, die nicht dominiert werden, wird Skyline genannt. Die Skyline wird ab einer bestimmten Menge an Kriterien zu groß und  erschwert somit die Entscheidung für den Kunden. 
In den folgenden Abschnitten und Kapiteln werden Produkte allgemein als Datensatz bezeichnet, da dies auch auf anderen Gebieten, wie der Suche des besten Basketballspielers übertragbar sind.
Genau genommen sind Datensätze Tupel einer Datenbank und die Spalten dieser Datensätze werden als \textit{Dimensionen} (z.B. Name des Spielers, Anzahl an Rebounds, etc.) bezeichnet.
%% ==============================
\section{Definitionen}
\label{ch:Einleitung:sec:Definitionen}
%% ==============================
Eine Skyline beinhaltet alle signifikanten Datensätze, die bezüglich der berücksichtigten Attribute von keinem anderen dominiert werden. Attribute für ein Auto sind, wie schon erwähnt, zum Beispiel der Preis oder die PS-Leistung. Um beurteilen zu können, ob ein Attribut eines Datensatzes besser als, das eines anderen ist, wird eine Präferenz des Users benötigt. Diese Präferenz gibt an, welche Attribute betrachtet und welche davon minimiert (geringer Preis) oder maximiert (hohe PS-Leistung) werden sollen. Für die Berechnung einer solchen Skyline gibt es schon viele effiziente Algorithmen, die zum Beispiel in \cite{borzsony2001skyline}, \cite{Chan:2006:HDS:2117976.2118017}, \cite{Kossmann:2002:SSS:1287369.1287394}, \cite{Papadias:2003:OPA:872757.872814} und \cite{Tan:2001:EPS:645927.672217} näher erläutert werden.

Um das Problem der zur großen Skyline zu lösen, wurde das Konzept der repräsentativen Skyline eingeführt. Diese enthält nur $k$ Datensätze der ursprünglichen Skyline und diese $k$ Datensätze sollen die Skyline so gut wie möglich repräsentieren. Diese Algorithmen geben hauptsächlich die unterschiedlichsten oder signifikantesten Datensätzen als Ergebnis aus, um damit die Skyline gut darstellen zu können. 
%% ==============================
\section{Problemstellung}
\label{ch:Einleitung:sec:Problemstellung}
%% ==============================
Eine Skyline kann durch ein Nested SQL Statement bestimmt werden. Dies ist jedoch nicht performant, was in \textit{Kapitel 3.1} von \cite{borzsony2001skyline} an einem Beispiel anschaulich erklärt wird. Repräsentative Skylines können somit meistens auch nicht sinnvoll durch SQL Statements identifiziert werden, was darauf schließen lässt, dass die verschiedenen Algorithmen zur Berechnung von repräsentativen Skylines \footnote{siehe \cite{Tao:2009:DRS:1546683.1547325}, \cite{cai2015efficient}, \cite{magnani2014taking} und \cite{4221657}} extern implementiert werden müssen, welches zusätzliche Rechenzeit kostet.

Weiterhin sollten Algorithmen für jeden nutzbar und so einfach bedienbar wie möglich sein. Der User sollte sich nicht mit den Details aufhalten, sondern lediglich ein paar Eingaben in einer graphischen Oberfläche machen. 
Ein großes Problem bei der Berechnung von (repräsentativen) Skylines liegt darin, dass keine nicht-numerischen Kriterien, wie zum Beispiel die Farbe eines Autos, berücksichtigen werden können. 
Für Probleme dieser Art wird im nächsten Abschnitt eine Lösung aufgezeigt.
%% ==============================
\section{Zielsetzung}
\label{ch:Einleitung:sec:Zielsetzung}
%% ==============================
Für die Berechnungen von Skylines und repräsentativen Skylines müssen Präferenzen für die einzelnen Kriterien, die in Betracht gezogen werden, vorhanden sein. Diese Präferenzen zeigen, welche Dimensionen minimiert oder maximiert werden sollen.
Eines der Ziele dieser Arbeit ist es, diese Präferenzen mit Hilfe von Preference SQL zu erweitern. Preference SQL wurde an der Universität Augsburg erforscht und entwickelt. \footnote{siehe \cite{kiessling2011preference}, \cite{kiessling2002foundations} und \cite{kiessling2002preference}}
\enquote{The Preference SQL System is implemented as a middleware component, enabling a seamless application integration with standard SQL backend systems.} \cite[p. 1]{kiessling2011preference}
Es bietet die Möglichkeit nicht nur Minimierung und Maximierungs Präferenzen zu berücksichtigen, sondern auch andere Präferenzen wie zum Beispiel AROUND oder LAYERED. Die AROUND Präferenz bevorzugt Werte, die nah an einem bestimmten Wert liegen. Im Gegensatz dazu kann der User bei der LAYERED Präferenz die Werte eines Kriteriums nach Wichtigkeit sortieren. Ein weiterer wichtiger Aspekt der LAYERED Präferenz ist, dass diese auch nicht-numerische Werte berücksichtigen kann. Dies führt dazu, dass zum Beispiel auch Autofarben bei Vergleichen zwischen Autos einbezogen werden können. Kapitel \ref{ch:Grundlagen:sec:präferenzen} beschäftigt sich mit allen Präferenzen, die für diese Arbeit wichtig sind und gibt einen kurzen Überblick über diese.
Die Einführung von Preference SQL bietet viele Vorteile und hilft dabei Skylinedatensätze besser zu bestimmen, aber auch im Bereich der Kundenberatung bringt es viele Vorteile Preference SQL anzuwenden.
\enquote{Its benefits comprise cooperative query answering and smart customer advice, leading to higher e-customer satisfaction and shorter development times of personalized search engines.}\cite[p. 1]{kiessling2002preference}

Für die Implementierung der Algorithmen, die in dieser Arbeit vorgestellt werden, wurde KNIME benutzt. KNIME ist eine Software für die interaktive und graphische Datenanalyse. Sie hilft dem User dabei diese Algorithmen ohne großes Verständnis benutzen zu können, da alle Implementierungsdetails dem User versteckt bleiben und mit Hilfe der graphischen Oberfläche die Idee der Algorithmen vereinfacht vermittelt. Die Algorithmen werden hierfür in einem KNIME Node implementiert, der vom Nutzer in einem Workflow erstellt werden kann. Die benötigten Daten für den implementierten Node, werden von anderen Nodes zu Verfügung gestellt. Danach kann der User den Node ausführen und die Ergebnisse betrachten. 
%% ==============================
\section{Aufbau der Arbeit}
\label{ch:Einleitung:sec:Gliederung}
%% ==============================
Dieses Kapitel hat einen kurzen Einblick in das Thema geliefert. Im nächsten Kapitel wird der Forschungsstand von repräsentativen Skylines und Preference SQL dargestellt. Einzelne Ansätze zur Berechnung von repräsentativen Skylines werden vorgestellt und unterteilt in verschiedene Bereiche basierend auf deren Verfahrensweise. 
Kapitel \ref{ch:Grundlagen} definiert wichtige Begriffe für (repräsentative) Skylines und erläutert deren Ziele und Nutzen. Zusätzlich erklärt dieses Kapitel wichtige Begriffe von KNIME und gibt einen kurzen Einblick in die Software.
Kapitel \ref{ch:Konzept} beschäftigt sich mit den Algorithmen, die in KNIME implementiert wurden. Hier wurde unterteilt in Skyline und repräsentative Skylinealgorithmen. Der Grund für die Implementierung des Skylinealgorithmus besteht darin, dass fast alle repräsentative Skylinealgorithmen eine Skyline als Input benötigen. Zusätzlich werden weitere wichtige Klassen von KNIME aufgezeigt, die für die Implementierung von Nodes erforderlich sind. Kapitel \ref{ch:Implementierung} listet alle implementierten Nodes, deren Nutzen und Implementierungsdetails auf.
Kapitel \ref{ch:useCase} soll einen Einblick geben wie alle implementierten KNIME Nodes funktionieren. Dies wird an einem Beispiel dargestellt, in dem das beste Auto in einer Menge von Datensätzen anhand bestimmter Kriterien gesucht wird. Im letzten Kapitel wird das Thema dieser Arbeit zusammengefasst und ein kurzer Überblick gegeben wie diese Arbeit fortgeführt wird.
%% ==============================
%%% End: 
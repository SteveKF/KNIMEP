%% Einleitung.tex
\chapter{Einleitung}
\label{ch:Einleitung}
%% ==============================
Die Suche nach einem bestimmten Produkt auf einer Internetseite liefert meistens zu viele oder gar keine Ergebnisse.  Dies liegt daran, dass Suchabfragen meistens auf harten Bedingungen beruhen, die den Wünschen des Kunden entsprechen. Eine harte Bedingung bedeutet, dass Produkte die Wünsche des Kunden erfüllen müssen und ansonsten nicht in der Ergebnismenge auftauchen. Dabei macht es keinen Unterschied, dass bei einem leeren Ergebnis Produkte existieren, die beinahe die Kriterien des Kunden erfüllen würden. Weiterhin werden keine Produkte aus der Ergebnisliste entfernt, die in allen ausgewählten Kriterien schlechter sind als andere. Dies führt zu großen Ergebnismengen, die verhindert werden können. Um diese beiden Effekte (keine/zu viele Ergebnisse), auf die später in der Arbeit noch genauer eingegangen werden, zu vermeiden, kann der User seine Wünsche in Präferenzen äußern. Präferenzen erlauben es dem User nicht nur seine eigenen Bevorzugungen in die Suche mit einfließen zu lassen, sondern zwingt eine reine Ausgabe der besten Ergebnisse. Präferenzen sorgen zusätzlich dafür, dass die Wünsche des Kunden nicht unbedingt erfüllt werden müssen, wenn dies nicht möglich ist. Stattdessen werden wie schon erwähnt die Produkte als Ergebnis ausgegeben, die die Wünsche des Kunden am besten erfüllen.
Bei Präferenzen ist es daher nicht nur wichtig, welche Attribute (Autobeispiel: Preis, PS) für den Nutzer am wichtigsten sind, sondern auch wie diese bevorzugt werden. Im Falle eines Autos kann als Beispiel angenommen werden, dass der Preis eines Autos so gering wie möglich sein sollte und die PS Anzahl so hoch wie möglich.
Die Produkte werden dann anhand dieser Präferenzen miteinander verglichen. Schlussendlich bleiben nur noch die besten, die von allen anderen undominiert sind, übrig. Die Dominanz ist hierbei wie folgt definiert: Ein Produkt dominiert ein anderes, wenn es in allen Kriterien mindestens genauso gut und in einem Kriterium besser ist. Ob ein Produkt in einem Kriterium besser ist als ein anderes, hängt von den Präferenzen ab. Da solche Vergleiche mit Präferenzen nur die besten ausgeben und die Wünsche des Kunden nicht unbedingt perfekt erfüllt sein müssen, ist zu erkennen, dass Präferenzen sich wie weiche Bedingungen verhalten.
Jedoch besteht ein Problem darin, dass umso mehr Kriterien für die Bewertung eines Produkts wichtig sind, desto schwieriger wird es sich für eines zu entscheiden, da dadurch mehr Produkte nicht mehr vergleichbar  sind und somit undominiert bleiben. Die Menge aller undominierten Produkte, die auch Skyline genannt wird, wird dadurch viel zu groß und erschwert damit die Entscheidung für den Kunden eines auszuwählen. 
In den folgenden Abschnitten und Kapitel werden Produkte allgemein als Datensatz bezeichnet, da dies auch auf andere Gebiete, wie die Suche des besten Basketballspielers übertragbar sind.
Genau genommen sind Datensätze Tupel einer Datenbank und die Spalten dieser Datensätze werden als \textit{Dimensionen} (z.B. Name des Spielers, Anzahl an Rebounds, etc.) bezeichnet.
\myworries{Mehr Begriffe, die für folgende Kapitel wichtig sind?}
%% ==============================
\section{Definitionen}
\label{ch:Einleitung:sec:Definitionen}
%% ==============================
Eine Skyline beinhaltet alle signifikanten Datensätze, die bezüglich der berücksichtigten Attribute von keinem anderen dominiert werden. Attribute für ein Auto sind wie schon erwähnt zum Beispiel der Preis oder die PS Leistung. Um beurteilen zu können, ob ein Attribut eines Datensatzes besser als das eines anderen ist, wird eine Präferenz des Users benötigt. Diese Präferenz gibt an, welche Attribute überhaupt betrachtet und welche davon minimiert (geringer Preis) oder maximiert (hohe PS Leistung) werden sollen. Für die Berechnung einer solchen Skyline gibt es schon viele effiziente Algorithmen, die zum Beispiel in \cite{borzsony2001skyline}, \cite{Chan:2006:HDS:2117976.2118017}, \cite{Kossmann:2002:SSS:1287369.1287394}, \cite{Papadias:2003:OPA:872757.872814} und \cite{Tan:2001:EPS:645927.672217} näher erläutert werden.

Um das Problem der zur großen Skyline zu lösen, wurde das Konzept der repräsentativen Skyline eingeführt. Diese enthält nur $k$ Datensätze der ursprünglichen Skyline und diese $k$ Datensätze sollen die Skyline so gut wie möglich repräsentieren. Diese Algorithmen geben hauptsächlich die unterschiedlichsten oder signifikantesten Datensätzen als Ergebnis aus, um damit die Skyline gut repräsentieren zu können. 
%% ==============================
\section{Problemstellung}
\label{ch:Einleitung:sec:Problemstellung}
%% ==============================
Eine Skyline kann durch ein Nested-SQL Statement bestimmt werden. Dies ist jedoch nicht performant, was in \textit{Kapitel 3.1} von \cite{borzsony2001skyline} an einem Beispiel anschaulich erklärt wird. Repräsentative Skylines können somit meistens auch nicht sinnvoll durch SQL Statements identifiziert werden und dies lässt darauf schließen, dass die verschiedenen Algorithmen zur Berechnung von repräsentativen Skylines (siehe \cite{Tao:2009:DRS:1546683.1547325}, \cite{cai2015efficient}, \cite{magnani2014taking} und \cite{4221657}) extern implementiert werden müssen und dies zusätzliche Rechenzeit kostet.

Weiterhin sollten Algorithmen für jeden nutzbar und so einfach bedienbar wie möglich sein. Der User sollte sich nicht mit den Details ärgern und sollte sich nur, um ein paar Eingaben in einer graphischen Oberfläche kümmern. 
Ein großes Problem bei der Berechnung von (repräsentativen) Skylines liegt darin, dass keine nicht-numerischen Kriterien, wie zum Beispiel die Farbe eines Autos, berücksichtigen werden können. 
Für diese Probleme werden im nächsten Abschnitt eine Lösung aufgezeigt.
%% ==============================
\section{Zielsetzung}
\label{ch:Einleitung:sec:Zielsetzung}
%% ==============================
Für die Berechnungen von Skylines und repräsentativen Skylines müssen Präferenzen für die einzelnen Kriterien, die in Betracht gezogen werden, vorhanden sein. Diese Präferenzen zeigen, welche Dimensionen minimiert oder maximiert werden sollen.
Eines der Ziele dieser Arbeit ist es, diese Präferenzen mit Hilfe von Preference-SQL zu erweitern. PreferenceSQL wurde an der Universität Augsburg erforscht und entwickelt (siehe \cite{kiessling2011preference}, \cite{kiessling2002foundations} und \cite{kiessling2002preference}). 
\enquote{The Preference SQL System is implemented as a middleware component, enabling a seamless application integration with standard SQL backend systems.} \cite[p. 1]{kiessling2011preference}
Es bietet die Möglichkeit nicht nur Minimierung und Maximierungs Präferenzen zu berücksichtigen, sondern auch andere Präferenzen wie zum Beispiel die AROUND oder LAYERED Präferenz. Die AROUND Präferenz bevorzugt Werte, die nah an einem bestimmten Wert liegen. Im Gegensatz dazu kann der User bei der LAYERED Präferenz die Werte eines Kriteriums nach Wichtigkeit sortieren. Ein weiterer wichtiger Aspekt der LAYERED Präferenz ist, dass diese auch nicht-numerische Werte berücksichtigen kann. Dies führt dazu, dass zum Beispiel auch Autofarben bei Vergleichen zwischen Autos einbezogen werden können. Kapitel \ref{ch:Grundlagen:sec:präferenzen} beschäftigt sich mit allen Präferenzen, die für diese Arbeit wichtig sind und gibt einen kurzen Überblick über diese.
Die Einführung von Preference-SQL bietet viele Vorteile und hilft dabei Skylinedatensätze besser zu bestimmen, aber auch im Bereich der Kundenberatung bringt es viele Vorteile Preference-SQL anzuwenden.
\enquote{Its benefits comprise cooperative query answering and smart customer advice, leading to higher e-customer satisfaction and shorter development times of personalized search engines.}\cite[p. 1]{kiessling2002preference}

Für die Implementierung der Algorithmen, die in dieser Arbeit vorgestellt werden, wurde KNIME benutzt. KNIME ist eine Software für die interaktive und graphische Datenanalyse. KNIME hilft dem User dabei diese Algorithmen ohne großes Verständnis benutzen zu können, da alle Implementierungsdetails dem User versteckt bleiben und mit Hilfe der graphischen Oberfläche die Idee der Algorithmen vereinfacht vermittelt. Die Algorithmen werden hierfür in einem KNIME Node implementiert, der vom Nutzer in einem Workflow erstellt werden kann. Die benötigten Daten für den implementierten Node, werden von anderen Nodes zu Verfügung gestellt. Danach kann der User den Node ausführen und die Ergebnisse betrachten. 
%% ==============================
\section{Gliederung/Aufbau der Arbeit}
\label{ch:Einleitung:sec:Gliederung}
%% ==============================
Dieses Kapitel hat einen kurzen Einblick in das Thema geliefert. Im nächsten Kapitel wird der Forschungsstand von repräsentativen Skylines und Preference-SQL dargestellt. Einzelne Ansätze zur Berechnung von repräsentativen Skylines werden vorgestellt und unterteilt in verschiedene Bereiche basierend auf deren Verfahrensweise. 
Kapitel \ref{ch:Grundlagen} definiert wichtige Begriffe für (repräsentative) Skylines und erläutert deren Ziele und Nutzen. Zusätzlich erklärt dieses Kapitel wichtige Begriffe von KNIME und gibt einen kurzen Einblick in die Software.
Das nächste Kapitel beschäftigt sich mit den Algorithmen, die in KNIME implementiert wurden. Hier wurde unterteilt in Skyline und repräsentative Skylinealgorithmen. Der Grund für die Implementierung des Skylinealgorithmus besteht darin, dass fast alle repräsentative Skylinealgorithmen eine Skyline als Input benötigen. Zusätzlich werden weitere wichtige Klassen von KNIME aufgezeigt, die für die Implementierung von Nodes erforderlich sind.
Kapitel \ref{ch:Evaluation} soll dem Leser einen Einblick geben wie alle implementierten KNIME Nodes funktionieren. Dies wird an einem Beispiel dargestellt, in dem das beste Auto in einer Menge von Datensätzen anhand bestimmter Kriterien gesucht wird. Im letzten Kapitel wird das Thema dieser Arbeit noch kurz zusammengefasst und ein kurzer Überblick gegeben wie diese Arbeit fortgeführt wird und werden kann.
%% ==============================
%%% End: 